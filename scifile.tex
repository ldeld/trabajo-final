\documentclass[letterpaper, 12pt]{article}


\usepackage{amsmath}
\usepackage{ragged2e} %para justificar el texto
\usepackage[font=small]{subcaption} %para capcionar subfiguras
\usepackage{graphicx} %para incluir imágenes
\usepackage[justification = centering, font=small, labelfont=bf]{caption} %cambia la talla de letra y posicion de las captions
\usepackage{pdfpages} %para incluir pdf
\usepackage{float}
\usepackage{listings} %para incluir código en c++
\usepackage{color}


\topmargin 0.0cm  %formatea las márgenes
\oddsidemargin 0.2cm
\textwidth 16cm 
\textheight 21cm
\footskip 1.0cm

%define colores cuando imprimes código
\definecolor{gray}{rgb}{0.5,0.5,0.5}
\definecolor{mauve}{rgb}{0.58,0,0.82}

\lstset{frame=tb,
  language=c++,
  aboveskip=3mm,
  belowskip=3mm,
  showstringspaces=false,
  columns=flexible,
  basicstyle={\small\ttfamily},
  numbers=none,
  numberstyle=\tiny\color{gray},
  keywordstyle=\color{blue},
  commentstyle=\color{red},
  stringstyle=\color{mauve},
  breaklines=true,
  breakatwhitespace=true,
  tabsize=3
}



%The next command sets up an environment for the abstract to your paper.

\newenvironment{sciabstract}{%
\begin{quote} \bf}
{\end{quote}}


% If your reference list includes text notes as well as references,
% include the following line; otherwise, comment it out.

\renewcommand\refname{References and Notes}

% The following lines set up an environment for the last note in the
% reference list, which commonly includes acknowledgments of funding,
% help, etc.  It's intended for users of BibTeX or the {thebibliography}
% environment.  Users who are hand-coding their references at the end
% using a list environment such as {enumerate} can simply add another
% item at the end, and it will be numbered automatically.

\newcounter{lastnote}
\newenvironment{scilastnote}{%
\setcounter{lastnote}{\value{enumiv}}%
\addtocounter{lastnote}{+1}%
\begin{list}%
{\arabic{lastnote}.}
{\setlength{\leftmargin}{.22in}}
{\setlength{\labelsep}{.5em}}}
{\end{list}}



\begin{document}


\large\title{{Sistema de Lorenz}} 
\author{L.M. Duque Valencia,$^{1}$ L. Del Castillo Detoeuf,$^{2}$ J.F. Reyes Botero$^{3}$}
\date{}
\maketitle
\centering\normalsize{$^{1, 2, 3}$Universidad Nacional de Colombia}\\
\centering\normalsize{Departamento de F\'isica, Facultad de Ciencias}\\
\centering\normalsize{Programaci\'on e Introducci\'on a los M\'etodos Num\'ericos}\\
\centering\normalsize{\{$^{1}$lmduquev, $^{2}$ldeld, $^{3}$jfreyesb\}@unal.edu.co\\

\baselineskip18pt



\begin{sciabstract}
  Este proyecto se propone estudiar gr\'aficamente el sistema de ecuaciones diferenciales conocido como sistema de Lorenz haciendo uso del m\'etodo de aproximaci\'on de Runge-Kutta 4 y de herramientas computacionales estudiadas durante el semestre. En primer lugar se model\'o el sistema usando un programa en lenguaje en C++, se graficaron las soluciones obtenidas en 3D y las proyecciones sobre los tres planos principales usando el programa Gnuplot. Finalmente, se modificaron ligeramente las condiciones iniciales del modelo para comprobar que se trata de un sistema ca\'otico.
\end{sciabstract}



% In setting up this template for *Science* papers, we've used both
% the \section* command and the \paragraph* command for topical
% divisions.  Which you use will of course depend on the type of paper
% you're writing.  Review Articles tend to have displayed headings, for
% which \section* is more appropriate; Research Articles, when they have
% formal topical divisions at all, tend to signal them with bold text
% that runs into the paragraph, for which \paragraph* is the right
% choice.  Either way, use the asterisk (*) modifier, as shown, to
% suppress numbering.

\section*{Introducci\'on}

\justify
El sistema de Lorenz, introducido por Edward Lorenz en 1963, es un modelo tridimensional de la din\'amica atmosf\'erica terrestre expresado en sistema no lineal y determinista de tres ecuaciones diferenciales. El modelo se basa en tres par\'ametros y las condiciones iniciales del sistema de ecuaciones: $\sigma$ (número de Prant), $\rho$ (número de Rayleigh) y $\beta$.
El sistema se expresa de la siguiente forma:
\begin{align}
\dot{X} = \sigma (Y-X)\\
\dot{Y} = X(\rho - Z)\\
\dot{Z} = XY - \beta Z
\end{align}
Donde $X$ es la velocidad de la corriente de convecci\'on, $Y$ es proporcional a la diferencia de temperatura entre  corrientes de convecci\'on ascendetes y descendentes y $Z$ es la desviaci\'on de la temperaratura respecto a la dependencia lineal de la altura. Este sistema es de gran inter\'es matem\'atico pues para ciertos valores de los par\'ametros muestra un comportamiento ca\'otico con un atractor conocido como "\textit{Mariposa de Lorenz}".\\
Para resolver este sistema de ecuaciones diferenciales, se utiliza el método de "\textit{Runge-Kutta 4}", el cual consiste en obtener la soluci\'on del problema inicial planteado, por medio de la aproximaci\'on en cada punto de la malla, bas\'andose en el resultado obtenido para el punto anterior. Teniendo como problema inicialpor ejemplo:
\begin{align}
  \frac{dy}{dt} = f(t,y)  &  t_0 < t < t_n\\
  y(0) = "\textit{c}"
\end{align}
Donde la malla está determinada por {t_0, t_1,... , t_n} de paso $h$, pasa obtener un resultado aproximado en estos puntos. \\
El método de RK4 es una generalizaci\'on de la f\'ormula b\'asica de Euler $y_{i+1} = y_i + h f(t_i, y_i)$, en los que los valores de f se reemplazan por un promedio ponderado de f en $t_i < t < t_{i+1}$; donde el orden del m\'etodo es la cantidad de t\'erminos que se usan en el promedio ponderado, en este caso será de orden 4.



\section*{Programaci\'on}

\justify
El programa implementado para aproximar las soluciones del sistema de Lorenz consiste en 5 funciones y una estructura, e imprime los valores sucesivos de $X$, $Y$, y $Z$ cada $10^{-4}$ segundos en un rango de tiempo de 0 a 50. En primer lugar, se declara una estructura con tres variables \textit{double} $x_0$, $y_0$, y $z_0$ que representan los valores de las ecuaciones en un tiempo $t$:

\begin{lstlisting}
struct ecuaciones {
  double x0;
  double y0;
  double z0;
} ecuacion;
\end{lstlisting}
 En seguida, implementamos tres funciones \textit{ff, gg} y \textit{hh}, una para cada ecuaci\'on diferencial del sistema de Lorenz. Usamos los siguientes valores para los par\'ametros del sistema: $\sigma = 10$, $\rho = \frac{8}{3}$, $\beta = 28$. 
\begin{lstlisting}
double ff(double xx, double yy, double zz) {
  double ff = 10 * (yy - xx);
  return ff;
}

double gg(double xx, double yy, double zz) {
  double gg = xx* (28 - zz) - yy;
  return gg;
}

double hh(double xx, double yy, double zz) {
  double hh = xx*yy - zz*(8.0/3.0);
  return hh;
}
\end{lstlisting}

Despu\'es introducimos una funci\'n que realice los c\'alculos del m\'etodo RK4. Esta funci\'on toma 4 argumentos: las variables $x_0$, $y_0$, y $z_0$ en un tiempo $t_i$ y un $\Delta t $ que llamamos \textit{step}. Esta funci\'on usa un total de 12 variables (los \textit{k, l,} y \textit{m} necesarios en el m\'etodo RK4) y llama a las tres funciones de las ecuaciones para aproximar los valores del sistema en el tiempo $t_{i+1} = t_i + \Delta t$. La funci\'on finaliza modificando los valores $x_0$, $y_0$, y $z_0$ de la estructura por los calculados con el m\'etodo RK4:
\begin{lstlisting}
//funcion que calcula valores de rk4
void rk4(double x0, double y0,double z0, double step) {

  double k0 = step * ff(x0, y0, z0);
  double l0 = step * gg(x0, y0, z0);
  double m0 = step * hh(x0, y0, z0);
  double k1 = step * ff(x0 + 0.5*k0, y0 + 0.5*l0, z0 + 0.5*m0);
  double l1 = step * gg(x0 + 0.5*k0, y0 + 0.5*l0, z0 + 0.5*m0);
  double m1 = step * hh(x0 + 0.5*k0, y0 + 0.5*l0, z0 + 0.5*m0);
  double k2 = step * ff(x0 + 0.5*k1, y0 + 0.5*l1, z0 + 0.5*m1);
  double l2 = step * gg(x0 + 0.5*k1, y0 + 0.5*l1, z0 + 0.5*m1);
  double m2 = step * hh(x0 + 0.5*k1, y0 + 0.5*l1, z0 + 0.5*m1);
  double k3 = step * ff(x0 + k2, y0 + l2, z0 + m2);
  double l3 = step * gg(x0 + k2, y0 + l2, z0 + m2);
  double m3 = step * hh(x0 + k2, y0 + l2, z0 + m2);

  ecuacion.x0 += (1./6)*(k0 + 2*k1 + 2*k2 + k3);
  ecuacion.y0 += (1./6)*(l0 + 2*l1 + 2*l2 + l3);
  ecuacion.z0 += (1./6)*(m0 + 2*m1 + 2*m2 + m3);
 
}
\end{lstlisting}

Finalmente, usamos la funci\'on \textit{main} para asignar los valores de las condiciones iniciales a las variables de la estructura y el valor de $\Delta t$, que tomamos como $10^{-6}$. Luego iteramos la funci\'on RK4 desde $t = 0$ a $t = 50$, con incremento $\Delta t$ (la variable \textit{step}), imprimiendo a cada paso los resultados de $x_0$, $y_0$, y $z_0$. Estos datos son guardados en un archivo de texto para ser graficados usando \textbf{Gnuplot}:
\begin{lstlisting}
int main(void) {
  //para modificar las condiciones iniciales, descomentar "+pow(10, 6)"
  
  ecuacion.x0 = 0;
  ecuacion.y0 = 1 //+pow(10, -6);
  ecuacion.z0 = 0;
  double step = 0.0001;
  
  for (double t=0 ; t < 50; t+=step) {
    rk4(ecuacion.x0, ecuacion.y0, ecuacion.z0, step);
    std::cout<< ecuacion.x0 << "\t" << ecuacion.y0 << ecuacion.z0 << "\n";

    //para imprimir datos de los planos (descomentar sólo una de las líneas):
    //std::cout << ecuacion.x0 << "\t" << ecuacion.y0 << "\n";
    //std::cout << ecuacion.x0 << "\t" << ecuacion.z0 << "\n";
    //std::cout << ecuacion.y0 << "\t" << ecuacion.z0 << "\n";

    //para imprimir Y en función de t
    //std::cout << t << "\t" << ecuacion.y0 << "\n"; 
  }  
  return 0;
}
\end{lstlisting}

Como se puede observar en los comentarios, para obtener los datos y graficar las proyecciones sobre los tres planos $XY$, $XZ$ y $YZ$, y la gr\'afica de $Y$ en funci\'on de $t$, usamos el mismo programa modificando \'unicamente la \'ultima l\'inea del ciclo \textit{for} para que imprima s\'olo los valores necesarios ($X$ y $Y$, $X$ y $Z$, $Y$ y $Z$, y $t$ y $Y$, respectivamente).


\section*{Graficar los resultados}

\justify
Al graficar los resultados en \textbf{Gnuplot}, s\'olo se usaron uno de cada 10 datos para que la talla de los archivos no fuera demasiado grande. Usamos el siguiente c\'odigo para generar los pdf:
\lstset{language =}
\begin{lstlisting}
gnuplot> set term pdfcairo
gnuplot> set output "lorenz.pdf"
gnuplot> set title "Sistema de Lorenz"
gnuplot> set xlabel "x"
gnuplot> set ylabel "y"
gnuplot> set zlabel "z"
gnuplot> splot "datos.txt" every 10 notitle pointtype 0 lt -1
gnuplot> set output
\end{lstlisting}

Obtuvimos la siguiente "Mariposa de Lorenz":
\clearpage

\begin{figure} [!htbp]
\centering
\includegraphics[scale=0.4]{imagenes/mariposa.jpg}
\caption{Gr\'afica tridimensional de las soluciones del sistema de Lorenz, conocidas como "Mariposa de Lorenz"}

\begin{subfigure}{0.31\textwidth}
\includegraphics[width=\linewidth ,height = 4cm]{imagenes/xy.jpg}
\caption{En el plano XY}
\end{subfigure}
\begin{subfigure}{0.31\textwidth}
\includegraphics[width=\linewidth ,height = 4cm]{imagenes/xz.jpg}
\caption{En el plano XZ}
\end{subfigure}
\begin{subfigure}{0.31\textwidth}
\includegraphics[width=\linewidth ,height = 4cm]{imagenes/yz.jpg}
\caption{En el plano YZ}
\end{subfigure}
\caption{Proyecciones en los tres planos}
\end{figure}


Podemos tambi\'en observar el comportamiento ca\'otico del sistema graficando las ecuaciones por aparte en funci\'on del tiempo y modificando solo una de las condiciones iniciales. 
\clearpage

\begin{figure} [t]
\centering
\includegraphics[scale = 0.6]{imagenes/yy.png}
\caption{Gr\'aficas de Y en funci\'on del tiempo. En negro se tiene la gr\'afica de la soluci\'on de Y usando las mismas condiciones iniciales que para la "Mariposa de Lorenz" ( $X(0) = 0, Y(0) = 1, Z(0) = 0$). En rojo, se tiene la soluci\'on al alterar ligeramente las condiciones iniciales: en este caso, tomamos $Y(0) = 1 + \epsilon$ con $\epsilon = 10^{-6}$. Las dos soluciones son muy similares al inicio, pero alrededor de $t = 30$ comienzan a desviarse y resultan siendo completamente diferentes, lo que muestra el comportamiento ca\'otico del sistema.}
\end{figure}


\justify
Para visualizar mejor la forma tridimensional de la "\textit{mariposa}", usamos la terminal de \textit{gif} de Gnuplot para animar la rotaci\'on sobre el eje $z$ de la gr\'afica (ver "mariposa.gif" en el repositorio). Usamos los siguientes comandos de Gnuplot para siular la rotaci\'on:

\begin{lstlisting}
gnuplot> set term gif animate
gnuplot> set output "Mariposa.gif"
gnuplot> set title "Mariposa de Lorenz"
gnuplot> set xlabel "x"
gnuplot> set ylabel "y"
gnuplot> set zlabel "z"
gnuplot> do for [a = 0 : 360] {
more> splot "datos.txt" every 10 notitle pointtype 0 lt -1
more> }
gnuplot> set output
\end{lstlisting}


\section*{Conlusiones}

\justify
Se diseñ\'o y desarroll\'o un programa en lenguaje C++ para resolver el sistema de Lorenz con las condiciones iniciales dadas. El programa fue escrito en el editor de texto "\textit{Emacs}" y compilado con "\textit{g++}", con la implementaci\'on de una estrucutura para el valor de las condiciones inciales, y el cual itera a través del método RK4, para la soluci\'on de un sistema de ecuaciones diferenciales, de 1 a 50 con un intervalo de $10^{-4}$. Una vez obtenidos las aproximaciones num\'ericas del sistema, se realiz\'o una gr\'afica por medio del programa "\textit{Gnuplot}", imprimiendo cada 10 datos, es decir 50000 datos en total, para que el progrma no se saturara de informaci\'on y pudiese correr de manera eficiente; para cada gr\'afica se tomaron los datos necesarios: gr\'afica tridimensional $x$ vs $y$ vs $z$ y  planos $x$ vs $y$, $x$ vs $z$, $y$ vs $z$. Luego, se grafic\'o el mismo sistema incial pero con una ligera modificaci\'on en una de las condiciones iniciales, tomado en este caso $Y(0)$, y con la gr\'afica de este respecto al tiempo $t$ se puede observar que con una mínima alteraci\'on en las condiciones inciales el sistema puede cambiar su comportamiento en una diferencia bastante notable, mostrando as\'i el comportamiento ca\'otico caracter\'istico del sistema. Adicionalmente, para complementar el trabajo realizado, por medio de "\textit{Gnuplot}" se implement\'o la rotaci\'on de la gráfica tridimensional sobre el eje $z$  obtenida anteriormente a manera de "\textit{gif}".


\section*{Formatting Citations}

Citations can be handled in one of three ways.  The most
straightforward (albeit labor-intensive) would be to hardwire your
citations into your \LaTeX\ source, as you would if you were using an
ordinary word processor.  Thus, your code might look something like
this:


\begin{quote}
\begin{verbatim}
However, this record of the solar nebula may have been
partly erased by the complex history of the meteorite
parent bodies, which includes collision-induced shock,
thermal metamorphism, and aqueous alteration
({\it 1, 2, 5--7\/}).
\end{verbatim}
\end{quote}


\noindent Compiled, the last two lines of the code above, of course, would give notecalls in {\it Science\/} style:

\begin{quote}
\ldots thermal metamorphism, and aqueous alteration ({\it 1, 2, 5--7\/}).
\end{quote}

Under the same logic, the author could set up his or her reference list as a simple enumeration,

\begin{quote}
\begin{verbatim}
{\bf References and Notes}

\begin{enumerate}
\item G. Gamow, {\it The Constitution of Atomic Nuclei
and Radioactivity\/} (Oxford Univ. Press, New York, 1931).
\item W. Heisenberg and W. Pauli, {\it Zeitschr.\ f.\ 
Physik\/} {\bf 56}, 1 (1929).
\end{enumerate}
\end{verbatim}
\end{quote}

\noindent yielding

\begin{quote}
{\bf References and Notes}

\begin{enumerate}
\item G. Gamow, {\it The Constitution of Atomic Nuclei and
Radioactivity\/} (Oxford Univ. Press, New York, 1931).
\item W. Heisenberg and W. Pauli, {\it Zeitschr.\ f.\ Physik} {\bf 56},
1 (1929).
\end{enumerate}
\end{quote}

That's not a solution that's likely to appeal to everyone, however ---
especially not to users of B{\small{IB}}\TeX\ \cite{inclme}.  If you
are a B{\small{IB}}\TeX\ user, we suggest that you use the
\texttt{Science.bst} bibliography style file and the
\texttt{scicite.sty} package, both of which we are downloadable from our author help site
(http://www.sciencemag.org/about/authors/prep/TeX\_help/).  You can also
generate your reference lists by using the list environment
\texttt{\{thebibliography\}} at the end of your source document; here
again, you may find the \texttt{scicite.sty} file useful.

Whether you use B{\small{IB}}\TeX\ or \texttt{\{thebibliography\}}, be
very careful about how you set up your in-text reference calls and
notecalls.  In particular, observe the following requirements:

\begin{enumerate}
\item Please follow the style for references outlined at our author
  help site and embodied in recent issues of {\it Science}.  Each
  citation number should refer to a single reference; please do not
  concatenate several references under a single number.
\item Please cite your references and notes in text {\it only\/} using
  the standard \LaTeX\ \verb+\cite+ command, not another command
  driven by outside macros.
\item Please separate multiple citations within a single \verb+\cite+
  command using commas only; there should be {\it no space\/}
  between reference keynames.  That is, if you are citing two
  papers whose bibliography keys are \texttt{keyname1} and
  \texttt{keyname2}, the in-text cite should read
  \verb+\cite{keyname1,keyname2}+, {\it not\/}
  \verb+\cite{keyname1, keyname2}+.
\end{enumerate}

\noindent Failure to follow these guidelines could lead
to the omission of the references in an accepted paper when the source
file is translated to Word via HTML.

\section*{Handling Math, Tables, and Figures}

Following are a few things to keep in mind in coding equations,
tables, and figures for submission to {\it Science}.

\paragraph*{In-line math.}  The utility that we use for converting
from \LaTeX\ to HTML handles in-line math relatively well.  It is best
to avoid using built-up fractions in in-line equations, and going for
the more boring ``slash'' presentation whenever possible --- that is,
for \verb+$a/b$+ (which comes out as $a/b$) rather than
\verb+$\frac{a}{b}$+ (which compiles as $\frac{a}{b}$).  Likewise,
HTML isn't tooled to handle certain overaccented special characters
in-line; for $\hat{\alpha}$ (coded \verb+$\hat{\alpha}$+), for
example, the HTML translation code will return [\^{}$(\alpha)$].
Don't drive yourself crazy --- but if it's possible to avoid such
constructs, please do so.  Please do not code arrays or matrices as
in-line math; display them instead.  And please keep your coding as
\TeX-y as possible --- avoid using specialized math macro packages
like \texttt{amstex.sty}.

\paragraph*{Displayed math.} Our HTML converter sets up \TeX\
displayed equations using nested HTML tables.  That works well for an
HTML presentation, but Word chokes when it comes across a nested
table in an HTML file.  We surmount that problem by simply cutting the
displayed equations out of the HTML before it's imported into Word,
and then replacing them in the Word document using either images or
equations generated by a Word equation editor.  Strictly speaking,
this procedure doesn't bear on how you should prepare your manuscript
--- although, for reasons best consigned to a note \cite{nattex}, we'd
prefer that you use native \TeX\ commands within displayed-math
environments, rather than \LaTeX\ sub-environments.

\paragraph*{Tables.}  The HTML converter that we use seems to handle
reasonably well simple tables generated using the \LaTeX\
\texttt{\{tabular\}} environment.  For very complicated tables, you
may want to consider generating them in a word processing program and
including them as a separate file.

\paragraph*{Figures.}  Figure callouts within the text should not be
in the form of \LaTeX\ references, but should simply be typed in ---
that is, \verb+(Fig. 1)+ rather than \verb+\ref{fig1}+.  For the
figures themselves, treatment can differ depending on whether the
manuscript is an initial submission or a final revision for acceptance
and publication.  For an initial submission and review copy, you can
use the \LaTeX\ \verb+{figure}+ environment and the
\verb+\includegraphics+ command to include your PostScript figures at
the end of the compiled PostScript file.  For the final revision,
however, the \verb+{figure}+ environment should {\it not\/} be used;
instead, the figure captions themselves should be typed in as regular
text at the end of the source file (an example is included here), and
the figures should be uploaded separately according to the Art
Department's instructions.


\section*{What to Send In}

What you should send to {\it Science\/} will depend on the stage your manuscript is in:

\begin{itemize}
\item {\bf Important:} If you're sending in the initial submission of
  your manuscript (that is, the copy for evaluation and peer review),
  please send in {\it only\/} a PostScript or PDF version of the
  compiled file (including figures).  Please do not send in the \TeX\ 
  source, \texttt{.sty}, \texttt{.bbl}, or other associated files with
  your initial submission.  (For more information, please see the
  instructions at our Web submission site,
  http://www.submit2science.org/ .)
\item When the time comes for you to send in your revised final
  manuscript (i.e., after peer review), we require that you include
  all source files and generated files in your upload.  Thus, if the
  name of your main source document is \texttt{ltxfile.tex}, you
  need to include:
\begin{itemize}
\item \texttt{ltxfile.tex}.
\item \texttt{ltxfile.aux}, the auxilliary file generated by the
  compilation.
\item A PostScript file (compiled using \texttt{dvips} or some other
  driver) of the \texttt{.dvi} file generated from
  \texttt{ltxfile.tex}, or a PDF file distilled from that
  PostScript.  You do not need to include the actual \texttt{.dvi}
  file in your upload.
\item From B{\small{IB}}\TeX\ users, your bibliography (\texttt{.bib})
  file, {\it and\/} the generated file \texttt{ltxfile.bbl} created
  when you run B{\small{IB}}\TeX.
\item Any additional \texttt{.sty} and \texttt{.bst} files called by
  the source code (though, for reasons noted earlier, we {\it
    strongly\/} discourage the use of such files beyond those
  mentioned in this document).
\end{itemize}
\end{itemize}

% Your references go at the end of the main text, and before the
% figures.  For this document we've used BibTeX, the .bib file
% scibib.bib, and the .bst file Science.bst.  The package scicite.sty
% was included to format the reference numbers according to *Science*
% style.


\bibliography{scibib}

\bibliographystyle{Science}




% Following is a new environment, {scilastnote}, that's defined in the
% preamble and that allows authors to add a reference at the end of the
% list that's not signaled in the text; such references are used in
% *Science* for acknowledgments of funding, help, etc.

\begin{scilastnote}
\item We've included in the template file \texttt{scifile.tex} a new
environment, \texttt{\{scilastnote\}}, that generates a numbered final
citation without a corresponding signal in the text.  This environment
can be used to generate a final numbered reference containing
acknowledgments, sources of funding, and the like, per {\it Science\/}
style.
\end{scilastnote}




% For your review copy (i.e., the file you initially send in for
% evaluation), you can use the {figure} environment and the
% \includegraphics command to stream your figures into the text, placing
% all figures at the end.  For the final, revised manuscript for
% acceptance and production, however, PostScript or other graphics
% should not be streamed into your compliled file.  Instead, set
% captions as simple paragraphs (with a \noindent tag), setting them
% off from the rest of the text with a \clearpage as shown  below, and
% submit figures as separate files according to the Art Department's
% instructions.


\clearpage

\noindent {\bf Fig. 1.} Please do not use figure environments to set
up your figures in the final (post-peer-review) draft, do not include graphics in your
source code, and do not cite figures in the text using \LaTeX\
\verb+\ref+ commands.  Instead, simply refer to the figure numbers in
the text per {\it Science\/} style, and include the list of captions at
the end of the document, coded as ordinary paragraphs as shown in the
\texttt{scifile.tex} template file.  Your actual figure files should
be submitted separately.



\end{document}











